\documentclass[12pt,a4paper,openright]{report}

\usepackage[english]{babel}
\usepackage{indentfirst}
\usepackage{xcolor}
\usepackage{eucal}
\usepackage{amsmath}
\usepackage{amsfonts}
\usepackage{comment}
\usepackage{newlfont}
\usepackage[parfill]{parskip}
\usepackage{listings} 
\usepackage{hyperref}
\usepackage{marvosym}
\usepackage{csquotes}
\usepackage{subfig}
\usepackage{adjustbox}
\usepackage{movie15}

\usepackage{algorithm}
\usepackage{algpseudocode}

% Biblatex - bibliograpy 
\usepackage[sorting=none]{biblatex}
\addbibresource{bibliography.bib}

% appendix 
\usepackage[toc,page]{appendix} 

% caption management
\usepackage[margin=1cm]{caption}

% some page formating
\usepackage[textwidth=400pt]{geometry}
\numberwithin{equation}{section} 
\linespread{1.3} 

% fancyhdr - Page format
\usepackage{fancyhdr}
\pagestyle{fancy}\addtolength{\headwidth}{20pt}
\renewcommand{\chaptermark}[1]{\markboth{\thechapter.\ #1}{}}
\renewcommand{\sectionmark}[1]{\markright{\thesection \ #1}{}}
\rhead[\fancyplain{}{\bfseries\leftmark}]{\fancyplain{}{\bfseries\thepage}}
\cfoot{}

% Tikz - Graph and diagrams
\usepackage{tikz}
\usetikzlibrary{arrows,positioning} 
\tikzset{
    %Define standard arrow tip
    >=stealth',
    %Define style for boxes
    punkt/.style={
           rectangle,
           rounded corners,
           draw=black, very thick,
           text width=6.5em,
           minimum height=2em,
           text centered},
    % Define arrow style
    pil/.style={
           ->,
           thick,
           shorten <=2pt,
           shorten >=2pt,}
}


% Json repr in listing
\colorlet{punct}{red!60!black}
\definecolor{background}{HTML}{EEEEEE}
\definecolor{delim}{RGB}{20,105,176}
\colorlet{numb}{magenta!60!black}

\lstdefinelanguage{json}{
    basicstyle=\tiny\ttfamily,
    numbers=left,
    numberstyle=\scriptsize,
    stepnumber=1,
    numbersep=8pt,
    showstringspaces=false,
    breaklines=true,
    frame=lines,
    backgroundcolor=\color{background},
    literate=
     *{0}{{{\color{numb}0}}}{1}
      {1}{{{\color{numb}1}}}{1}
      {2}{{{\color{numb}2}}}{1}
      {3}{{{\color{numb}3}}}{1}
      {4}{{{\color{numb}4}}}{1}
      {5}{{{\color{numb}5}}}{1}
      {6}{{{\color{numb}6}}}{1}
      {7}{{{\color{numb}7}}}{1}
      {8}{{{\color{numb}8}}}{1}
      {9}{{{\color{numb}9}}}{1}
      {:}{{{\color{punct}{:}}}}{1}
      {,}{{{\color{punct}{,}}}}{1}
      {\{}{{{\color{delim}{\{}}}}{1}
      {\}}{{{\color{delim}{\}}}}}{1}
      {[}{{{\color{delim}{[}}}}{1}
      {]}{{{\color{delim}{]}}}}{1},
}


% ragged2e - Justifying text
\usepackage{ragged2e}
\justifying


\begin{document} 


\begin{titlepage}
    \textwidth=400pt
    \begin{center}
    {{\large{\textsc{Alma Mater Studiorum $\cdot$ Universit\`a di
    Bologna}}}} 
    \rule[0.1cm]{400pt}{0.1mm}
    \rule[0.5cm]{400pt}{0.6mm}
    {\small{\bf DEPARTMENT OF ECONOMICS\\
        Second-cycle $\cdot$ Master's Degree \\ in \\ ECONOMICS }}
    \end{center}
    \vspace{15mm}
    \begin{center}
    {\Large{\bf A STUDY ON TAXATION POLICIES}}\\
    \vspace{3mm}
    {\Large{\bf   ON HETEROGENEOUS }}\\
    \vspace{3mm}
    {\Large{\bf  AI-DRIVEN AGENTS}}\\
    \end{center}
    \vspace{40mm}
    \par
    \noindent
    \begin{minipage}[t]{0.47\textwidth}
    {\large{\bf Defended by:\\
    LORENZO MEZZINI\\
    926012}} 
    \end{minipage}
    \hfill
    \begin{minipage}[t]{0.47\textwidth}\raggedleft
    {\large{\bf Supervisor:\\
    VINCENZO DENICOL\`O}}
    \end{minipage}
    \vspace{20mm}
    \begin{center}
    {\large{\bf Graduation Session of December\\
    Academic Year 2020/2021 }}
    \end{center}
\end{titlepage}


\begin{comment}

%------------------------------------------------------------------------------------
% TITLE PAGE
%------------------------------------------------------------------------------------
\begin{titlepage} 
\thispagestyle{empty}                   %elimina il numero della pagina
\topmargin=6.5cm                        %imposta il margina superiore a 6.5cm
\raggedleft                             %incolonna la scrittura a destra
\large                                  %aumenta la grandezza del carattere
\em                                     %emfatizza (corsivo) il carattere
Questa è la \textsc{Dedica}:\\
ognuno può scrivere quello che vuole, \\
anche nulla \ldots                   
\newpage                               
%\clearpage{\pagestyle{empty}\cleardoublepage}%non numera l'ultima pagina sinistra
\end{titlepage}
%------------------------------------------------------------------------------------
       
\end{comment}



%------------------------------------------------------------------------------------
% INTRO - TABLE OF CONTENTS
%------------------------------------------------------------------------------------
\pagenumbering{roman}                  
\chapter*{Abstract}             
\addcontentsline{toc}{chapter}{Abstract}

Throughout the thesis, it will be presented a partial paper reproduction, followed by an extension. This work presents a set of economical simulations with four heterogeneous AI-driven agents. This is done by the use of a python package called ai-economist\cite{aie-pypi} and techniques from reinforcement learning. The training of the agents is going to be divided into two steps, a first phase where the agents get accustomed to the environment, and a second one where different kinds of taxation will be imposed. The main focus will be the relationship between total production and equality. Quite reasonably, we found that there is a negative impact from taxes on productivity, and a positive impact on equality.

\clearpage{\pagestyle{empty}\cleardoublepage}
\tableofcontents      
\clearpage{\pagestyle{empty}\cleardoublepage}

\begingroup
\let\clearpage\relax
\listoffigures   
\listoftables  
\listofalgorithms
\endgroup

\clearpage{\pagestyle{empty}\cleardoublepage}

\pagenumbering{arabic}
%------------------------------------------------------------------------------------
\chapter*{Introduction}             
\addcontentsline{toc}{chapter}{Introduction}

One of the challenges in macroeconomic theory is to transpose the theory to the real world. In particular the process of designing a policy and implementing it. This is because the analytical solutions are bound to certain a level of complexity. The more complex and similar to reality the problem gets the harder it gets to solve it analytically. In addition agents in the real world are heterogeneous and thus each one is acting differently according to his skill and initial conditions. 

In this thesis, I am going to present a partial reproduction with an expansion of the paper \textit{The AI Economist: Optimal Economic Policy Design via Two-level Deep Reinforcement Learning}\cite{zheng2021ai}. With the scope of presenting an AI-driven economical simulation, where is possible to investigate a high complexity setup that would be almost impossible to study analytically. This simulation requires the agents to behave in a complex way. They are asked to maximize their utility in a simulated economy where they can gather goods, use them or interact with other agents in a bid/ask market. To achieve this goal we will use a technique called proximal policy optimization. This is an optimization algorithm that will train a fully connected neural network. It will be shown that AI-driven agents can find a sub-optimal behavior, and come up with emerging behaviors and specialization to maximize their utility.

Afterward, multiple kinds of taxation will be introduced in the simulations as a discriminant factor, and we will compare the results on total economical productivity and coin equality among agents. This requires a second training that will get the agents accustomed to the new setup. 


The paper that is used as a reference \cite{zheng2021ai}, addresses the question of whether is it possible to generate optimal policies through the use of reinforcement learning. To do it, they first built a gather-and-trade game, publicly available on GitHub, as a framework for the simulations. Afterward, they trained the agents to behave optimally within this simulation. Once they had this setup they created four simulations: one that recreates the US taxes, the free market scenario, the Saez taxation, and another RL driven policymaker that optimize the production of equality and productivity. They were able to show that the RL agent produced an economy the 16\% more efficient compared with the analytical solution purposed by Saez\cite{saez2001using}, in the variable of interest. The entirety of the code from this second part is not available since their paper is still in peer review.

Thus, using third-party code available on Github, I will try to reproduce the first step of optimization, where the agents maximise their utility. Then I will introduce four taxations as well, however, two of these will be the US and the free market, and the other two will be the Italian system and communism. Hence, the main questions that this thesis address are: Is reinforcement learning a viable media to construct an economical accurate simulation? Can we conclude the impact of the different taxation systems on the overall economy?

%------------------------------------------------------------------------------------
% CHAPTER 1
%------------------------------------------------------------------------------------
\chapter{What is Foundation}
In this chapter I am going to introduce the package provided by Salesforce called Foundation. This package offers the possibility to create simulations with multiple agents that interacts in a 2D world. First a presentation of the the Gather and Trade simulation setup will be purposed, followed by a description of the agents involved, their action set, observation and scope.

\textcolor{red}{descrivi il capitolo quando lo hai finito}

\section{Gather and trade}

The simulation that will be used throght the dissertation is called gather and trade. This simulation takes place in a 2D map that represent the world where the agents lives and interact. The shape of the world is a 25x25 grid where are disposed various kinds of entities. Within this world 4 agents are free to move around, gather resources, trade them and use them to build houses. These agents are different for their skill level, allowing them to have higher/lower rewards for their actions. A fifth agent, called policy maker, is asked to tax and redistribute the income of the 4 previous agents, based on informations about their endowments, but not their skill.

\subsection{World and entities}

As said before the map is a 25x25 grid that where are present some entities. Some of these are visible in the world, others are just present in the agents endowment.

\begin{itemize}
    \item Water
    \item House
    \item Source Block (Wood and Stone)
    \item Wood and Stone
    \item Coins
    \item Labor
\end{itemize}


Water is a simple world block that has no use other than avoiding agents to pass through. We can see from Figure \ref{img:map_0} that the water is used to divide the world in 4 macro areas, each one with different resources. 
A House is a block that is not present at the beginning of the simulation, but it can be built by agents in any empty block, agents can't walk through houses either.

Sorce blocks are two entities that spawns stochastically resources, namely wood and stone, as we can see from Figure \ref{img:map_0} in the four areas divided by water we have a zone with both wood and stone source block, two other areas with just one kind of source block and the last one that is empty. 

Coins is the measurement of the value produced in the world. Coins are generated when a house is built, the agent that builds the house is rewarded with a certain amount of coins that variates with the skill level.

Labor is a measurement of the total effort exherted by an agent, this is generated every time an agent takes an actions and generates disutility.


\begin{figure}[h!]
    \centering
    \linespread{.9} 
    \includegraphics[width=0.5\textwidth]{Resources/imgs/Map_0.png}
    \caption[Rendering of the world at inital conditions: ]%
    {\label{img:map_0}Rendering of the world at inital conditions: \small \textit{this is a rendering of the world map at the first timestep of the simulation, blue block are water, brown blocks are wood source, grey block are Stone source. The four corners are the starting positions for the 4 agents.}}
\end{figure}



\subsection{Game dynamics}

A problem that has a continuous flow of agent-environment interactons can be formalized by a Finite Markov Decision problem \cite{sutton2018reinforcement}. In particular this simulation is a partial-observable multi-agent Markov Games (MGs). The problem is defined by the tuple \(\left(  S,A,R, \text{\textcolor{red}{simbolo}} ,\gamma, o, \mathcal{T} \right) \) where \( S \) is the state space and \( A \) is the action space. 

The simulation is composed by a series of episodes, each of length \( H \) timesteps. At every point in time \( t \in [0,H] \) the episode is characterized by a state \( s_t \), representing the current world environment, every agent performs an action \( a_{i,t}  \)  given the current partial observation \( o_{i,t}\) of the world state, and recives a reward \( r_{it}  \). Afterwards the environment transits to a new state \( s_{t+1} \) according to the transition distribution \(  \mathcal{T}(s_{t+1}|s_t,\boldsymbol{a}_t)\). This chain of interactions state-action-reward/state carries on until the end of an episode.

\begin{equation*}
     s_0 \,  \rightarrow_{\boldsymbol{o}_0}\, \boldsymbol{a}_0 \,\rightarrow\, r_1 , s_1 \,\rightarrow_{\boldsymbol{o}_1}\, \boldsymbol{a}_1 \,\rightarrow ... \rightarrow_{\boldsymbol{o}_{H-1}}\, \boldsymbol{a}_{H-1} \,\rightarrow\, r_H , s_H
\end{equation*}

Here \( \boldsymbol{a} \) and \( \boldsymbol{o} \) are the vectorized observations and actions for all the five agents. Given the particular structure of the simulation every single agent will recive an observation at every timestep (different for everyone, more on that later), but only at the 4 basic agents will be asked to perform an action, the policy maker will act only upon a certain condition met. In this case the episode last for 1000 timestep and the policy maker is asked to act (tax the other agents) every other 100 steps. The existence of multiple episodes is necessary for the 4 agents and the policy maker to define their own optimal policy \( \pi_i(o_{i,t}, h_{i,t-1};\theta_i^*) \), this optimization process will be the focus of chapter \textcolor{red}{RL}.

\subsection{Agents}

From what above we know that the four basic agents are endowed with labor, coins, wood and stone. They live in the world map, can act within it and their objective is to maximize their $\gamma$-discounted utility function. Now I will describe in more in detail the agents starting from the inforamtions that they receive at each timestep, then talking about the actions that they are allowed to take and finally about their objective. 

\paragraph{Observation space:} Given that this simulation is a partial-observable multi-agent Markov Game, the observation that agent \( i \) receive at time \( t \) is not complete but partial, this inforamtions can be summarized in the following way:

\begin{itemize}
    \item \(o_{i,t}^{\text{world state}}\): world map situation surrounding the agent, this is limited to the 11x11 grid around the agent \( i \).
    \item \(o_{i,t}^{\text{market state}}\): full information about the market state for wood, stone and available trades.
    \item \(o_{i,t}^{\text{agent}}\): public resouces and coin endowments (this information is also available to the policy maker) and private labor performed and skill level.
    \item \( o_{i,t}^{\text{tax}} \): tax structure
    \item \( o_{i,t}^{\text{other}} \): other inforamtions (ex. action mask)
\end{itemize}
 
the full observation space can be seen in Table \ref{tab:full_obs}

\paragraph{Action space:} The agent can take one action per timestep and can choose this action from the 50 listed below:

\begin{itemize}
    \item \textbf{Movement}: 4 actions for the basic movements N, S, E, W
    \item \textbf{Gather}: 1 action for gahtering
    \item \textbf{Trade}: 44 actions for trading resouces
    \item \textbf{Idle}: 1 action that do notthing
\end{itemize}

The movements actions along with gather do not need much of an explaination, these are simple actions that costs a quantity of 0.21 labor units each time pursued. The building action require the agent to consume (destroy) one unit of wood and one unit of Stone, as a consequence he gains 2.1 units of labor and an amount of coin that depends on his skill level. The most complicate set of actions are the one that rules trading. Each one of them is a combination of the 11 price levels [0,1,...,10] that the agent is willing to (pay/request) for each side (bid/ask) and for each resource (wood/stone). A trading open action remains open for 50 tunrs, if in this time span it is matched by the corresponding counter action at the same price (a bid for an ask and vice versa) then the trade takes place and each agent gets a 0.05 units of labor.

The action mask, present in the observation space, is a touple of binary values of length 50 that "masks" inconclusive actions, such as moving north while at the north border of the map, or building a house without the required wood and stone. This is used in the learning process to avoid wasting time in exploring meaningless actions.

\paragraph{Agent objective}

Agents in the simulation earn coins when building houses or trading goods, 

The utility for the four agents is an isoelastic utility:

\begin{equation}
u_i(x_{i,t}, l_{i,t}) = crra(x_{i,t}^c) - \vartheta_k l_{i,t}\,, \quad crra(z) = \frac{z^{1- \eta}-1}{1-\eta}\,,\,\, \eta > 0
\end{equation}

Where \( l_{i,t} \) is the cumulative labor associated with the actions taken up to time \( t \), \( x_{i,t}^c \) is the coin endowment and \( \vartheta \) is a function utilized for labor anihilation with the following specification \( \vartheta_k = 1- exp\left(- \frac{\textit{episode completitions}}{\textit{energy warmup constant (k)}}\right)\). This variable will play an important role during the two step optimization process purposed in the original paper. In particular during the phase 1 of training the labor cost is annihilated to help agents avoid sub-optimal behaviours. And \( \eta \) determies the degree of nonlinearity of the utility. This utility function is assumed to be the same for all the agents.

The maximization problem is solved for a rational behaving agent by optimizing the total discounted utility over time,

\begin{equation}
\forall i \,:\, \max_{\pi_i}\mathbb{E}_{a_i \sim \pi_i, \boldsymbol{a}_{-i} \sim \boldsymbol{\pi}_{-i}, s^{'}\sim\mathcal{T}}\left[ \sum_{t=1}^H \gamma^t r_{i,t} + u_i({x_{i,0}l_{i,0}})\right]
\label{eq:agent_max}
\end{equation}


with \( r_{i,t} = u_i(x_{i,t},l_{i,t})  - u_i(x_{i,t-1},l_{i,t-1}) \) being the istantaneous reward of agent \( i \) at time \( t \). Equation \ref{eq:agent_max} illustrates a multi-agent optimization problem in which actors optimize their behavior at the same time, since the utility of each agent is dependent on the behavior of other agents. Another agent, for example, may deny an agent access to resources, limiting how many houses the agent can create in the future and hence its utility. While computing equilibria for complicated environments like this is still out of reach, we will see in \textcolor{red}{RL chapter} how RL may be utilized to produce meaningful, emergent behaviors.


\subsection{Policy maker}

The policy maker, or social planner, differs deeply from the previous agents. Being the focus of the research question it's structure and behavior changes a lot in every single simulaion. 

\paragraph{Observation space:} The observation space of the social planner depends on the simulation, for most of the simulations no observation is needed. 

\textcolor{red}{se riesci a fare RL anche su di lui descrivi l'obs space di quella simulazione}

\paragraph{Action space:} the action space is quite similar amongs all the simulations, the social planner has to decide how much to tax the individuals according to their total income. If the policy maker is 

\textcolor{red}{non è facile finchè non ho deciso le simulazioni da fare ... magari lo rimado a dopo}



%------------------------------------------------------------------------------------



%------------------------------------------------------------------------------------
% CHAPTER 2
%------------------------------------------------------------------------------------
\chapter{Reinforcement Learning}
Reinforcement learning is the process of learning what to do to maximize a numerical reward. A mechanism where the learner is not told what action to take, but instead must discover which action yields the highest reward by trying them. In this case, actions might affect not only the immediate reward but also future situations and all the subsequent rewards. These two characteristics -- Trial-and-error search and delayed reward -- are the two most important distinguishing features of reinforcement learning.

Reinforcement learning differs from supervised learning since training is not based on an external dataset of labeled examples, where each situation (observation) is labeled with the correct action to perform (often identify a category). RL, although one might erroneously think the opposite, is also different from unsupervised learning. The main objective for unsupervised learning is to find hidden structures in an unlabeled dataset, whereas RL's main objective is to maximize a reward signal. 

In this chapter, we will see the definition of a Markov decision process and those of the state and action value function. Afterward, we will observe how optimization is carried out in an approximate solution method, in particular in the case of the proximal policy optimization since it is the technique utilized for training the neural network.

\section{Finite Markov Decision process (MDPs)}

Finite Markov decision processes are a class of problems that formalize subsequent decision making, where not only the influence of the action is exerted on immediate reward but also on those in the future. MDP's are suited for RL since they frame the process of learning through repeated interaction between an agent (decision maker), and an environment (ruled by fixed state transition function).

More specifically an agent is asked to take an action \( a_t \in \mathcal{A} \) at every time step \( t = 0,1,...\, \). To do so the agent is provided with an observation of the current environment's state \( s_t \in \mathcal{S} \) and a reward \( r_t \in R \) from the previously performed action. Afterwards the environment update it
s state following a transition distribution \( \mathcal{T}(s_{t+1}|a_t,s_t) \) and a numerical reward \( r_{t+1} \in \mathcal{R} \subset \mathbb{R} \). This process is reproduced every subsequent timestep, this concatenation of interaction is a MDP.

\begin{figure}[!h]
  \centering
  \linespread{.9}
  \begin{tikzpicture}[node distance=1cm, auto,]
      \node[punkt] (agent) {Agent};
      \node[punkt, inner sep=5pt,below=1cm of agent] (env) {Environment};
      \path[draw,->] 
        (env.west) -- ++(0cm,.1cm)  -- ++(-1cm,0cm) -- ++(0,1.65cm)  -- ++(1cm,0cm);
      \path[draw,->] 
        (env.west) -- ++(0cm,-.1cm) -- ++(-1.6cm,0cm) -- ++(0,2.05cm) -- ++(1.6cm,0cm);

      \path[draw,->] 
        (agent.east) -- ++(1.25cm,0cm) -- ++(0,-1.85cm) -- (env.east);


      \draw (3.1,-1) node[] {\( a_{t} \)};
      \draw (-2,-1) node[] {\( r_{t+1} \)};
      \draw (-3.5,-1) node[] {\( s_{t+1} \)};
  \end{tikzpicture}
  \caption[Agent enviroment interaction dynamic: ]%
  {\label{img:a-e_dynamic}Agent enviroment interaction dynamic: \small \textit{this is the basic dynamic of the interaction between the agent and the enviroment in a MDP, at time t the agent provides an action a and the enviroment respond with a state s and rewad r that are used from the agent to decide the action at t+1 and so on.}}
\end{figure}


\paragraph{Objective and Rewads:} The main objective of RL is to maximize the total number of rewars it receive. This reward \( r_t \) is passed from the environment to the agent at every timestep as a consequence of his actions. In the case of the Gather and Trade the reward is \( r_{i,t} = u_i(x_{i,t},l_{i,t})  - u_i(x_{i,t-1},l_{i,t-1}) \). 

Since the agents want to maximize the total upcoming reward we can write this value simply as the sum of all the future rewards.

\begin{equation*}
U_t \doteq r_{t+1} + r_{t+2} + ... + r_{H},
\end{equation*}

Where \( H \) is the total length of the episode, and at the time step \( t = H \) the episode ends. This state is called the terminal state and is a particular state because regardless of the final condition of the agent it reset the environment to the initial condition and restart completely the episode.
Another specification for the total reward can implement the concept of discounting, which is more appropriate in the case of economical simulations, thus within the experiments the agent has to maximize his future discounted utility:

\begin{equation}
U_t \doteq r_{t+1} + \gamma r_{t+2} + \gamma^2 r_{t+3} +... + \gamma^{H-t-1}r_{H},
\end{equation}

the discount rate \( \gamma \in [0,1] \) determines the value that the agent assigns to the future reward, a reward received at \( k \) timesteps in the future is only valued \(\gamma^{k-1} \) times what it would be valued today. When the value of \( \gamma \) approaches 0 the agent is more "myopic" and puts most of his interest in immediate rewards, while if it approaches 1 the interest is more projected in the future due to the stronger impact of future rewards.

\paragraph{Value functions: } One of the most important elements of RL is the value function. The estimation of this function is one of the crucial points of RL, in fact it tries to quantify the expected return of the rewards it expects to receive. Furthermore, the expected reward depends on the action that the agent decides to take, thus the value function are defined in terms of policies, which are acting behaviors.

If the agent is following policy \( \pi \) at time \( t \), then \( \pi(a|s) \) is the probability that the agent takes the action \( a_t = a \) given the state \( s_t = s \). The aim of RL is to change the policy based on experience across episodes to find an optimal behavior.

We can write a value funciton for a state \( s \) under the policy \( \pi \). This function is the expected return when the inital state is \( s \) and the policy followed is \( \pi \).

\begin{equation}
v_\pi(s) \doteq \mathbb{E} _\pi \left[ U_t | s_t = s \right] \, =\,  \mathbb{E} _\pi \left[\left.\sum_{k = 0}^{H}\gamma^k r_{t+k+1}  \right| s_t = s \right], \quad \text{for all}\, s \in \mathcal{S} 
\label{eq:statevalue_function}
\end{equation}

\(  v_\pi(s)\) is called the state-value function for policy \( \pi \). Following from this equation, it is possible to define the value of taking an action \( a \) in the state \( s \) following the policy \( \pi \):

\begin{equation}
    q_\pi(s,a) \doteq \mathbb{E} _\pi \left[ U_t | s_t = s, a_t = a \right] \, =\,  \mathbb{E} _\pi \left[\left.\sum_{k = 0}^{H}\gamma^k r_{t+k+1}  \right| s_t = s, a_t = a \right],
    \label{eq:actionvalue_function}
\end{equation}
    
and \( q_\pi(s,a) \) is called the action-value function for policy \( \pi \). The important concept here is that the value functions in \ref{eq:statevalue_function} and \ref{eq:actionvalue_function} can be estimated from experience. 

There are multiple strategies to determine an optimal behavior starting from the evaluation of these functions, we can divide these ways in two main groups, tabular solution methods and approximate solution methods\cite{sutton2018reinforcement}. For the former we have Montecarlo methods, Dynamic programming, Temporal-difference learing, n-step bootsrap and others. While for the latter we have On/Off-policy methods with approximation and policy gradient methods. 

For the purpose of this thesis we are going to foucs only on policy gradient methods and a particular set of optimization policy called proximal policy optimization.

\section{Approximate solution Methods}

The approximate solution methods are sets of strategies thought for those problems, such as ours, where the set of possible states is enormous. It is very likely that every state encountered in a simulation will never have been encountered before. Thus, to make meaningful decisions there is the need to be able to generalize from previous state that are, to some extent, similar. This is accomplished by borrowing the exising generalization theory, usually by using the process of function aproximation. However, we are going to use policy gradient methods that do not necessarily need to estimate a value function through function aproximation. 



\subsection{Policy gradient Methods}  


Policy gradient methods are a set of parameterized policies that can select actions without the use of a value function. The method consists in calculating the estimator of the policy gradient and feeding it into a stochastic gradient ascent algorithm. We denote with \( \pi_\theta(a|s) \) the stochastic policy such that 
\begin{equation}
  \pi_\theta(a|s) = \pi(a| s, \mathbf{\theta}) = Pr \left\{ a_t = a | s_t = s \, , \,   \mathbf{\theta}_t =  \mathbf{\theta}\right\}
\end{equation}
for the probability of choosing action \( a \) at time \(  t \) given the state \( s_t \) and the parameters \(  \mathbf{\theta} \). The aim of the policy gradient methods is to maximize a perfomance measure \( J_t( \mathbf{\theta}) \). This is done by updating the parameters \(  \mathbf{\theta} \) with a gradient of the performance itself:

\begin{equation}
  \mathbf{\theta}_{t+1} =  \mathbf{\theta}_t + \alpha\widehat{\nabla J ( \mathbf{\theta}_t)}
\end{equation}



\subsection{Proximal Policy Optimization Algorithms}

Proximal policy optimization algorithms (PPOs) are a family of policy gradient methods for reinforcement learning, which alternate between sampling data through interaction with the environment, and optimizing a "surrogate" objective function using stochastic gradient ascent. 

PPO utilizes as a performance measure the advantage \( A(s,a) = q(s,a) - v(s) \), which represents how good an action is compared to the average action in a specific state.

It also relies on the ratio of the policy that we want to refine to the older policy:

\begin{equation}
r_t(\theta)= \frac{\pi_\theta(a_t|s_t)}{\pi_{\theta old}(a_t|s_t)}
\end{equation}

The surrogate objective function that we optimize with PPO is 

\begin{equation}
\mathcal{L}^{CLIP} (\theta) = \hat E_t \left[ min(r_t(\theta)\hat A_t \,, \, clip(r_t(\theta) \,,\, 1-\epsilon\,,\, 1+ \epsilon)\hat A_t)\right]
\end{equation}

The maximization of this surrogate funcition takes the advantages of trust region policy optimization (TRPO) \cite{schulman2015high}, adjusting the parameters in the dierction of the ratio \( r_t(\theta) \), and maintaining this change bounded in a "clipped" region to avoid drastic refinements of the policy. This is shown well in the Figure \ref{img:l_clip}


\begin{figure}[h!]
  \centering
  \linespread{.9} 
  \includegraphics[width=0.75\textwidth]{Resources/imgs/L_clip.PNG}
  \caption[One step \( L^{CLIP} \) as a function of \( r_t(\theta) \):  ]%
  {\label{img:l_clip}One step \( L^{CLIP} \) as a function of \( r_t(\theta) \): \small \textit{this is the function \( L^{CLIP} \) as a function of the probability ratio \( r_t(\theta) \) when the advantage A is positive or negative. The red dot is the starting point of the optimization. (Figure from \cite{schulman2017proximal})}}
\end{figure}


The surrogate function can be further augmented by adding an entropy bonus to ensure sufficient exploration. By adding these terms, we obtain the following objective function:

\begin{equation}
L_t^{CLIP+VF+S}(\theta) = \hat E_t \left[ L_t^{CLIP}(\theta) - c_1 L_t^{VF}(\theta)  + c_2 S[\pi_\theta](s_t)\right]
\end{equation}

where \( S \) in an entropy bonus, \(  L_t^{VF} \) is a squared-error loss and \(c_1, c_2 \) are coefficients.

The pseudocode below better explains all the procedure done by the PPO in refining the parameters.

\begin{algorithm}
  \caption{PPO with clipped objective}\label{alg:ppo_clip}
  \begin{algorithmic}
    \State Input: inital policy \( \theta_0 \), clipping threshold \( \epsilon \)
    \For{\( k = 0,1,2... \)}
      \State Collect a set of trajectories $\mathcal{D}_k$ on policy \( \pi_k = \pi(\theta_k) \)
      \State Estimate the adventages \( \hat A_t^{\pi_k} \), using any advantage est. alg. 
      
      \Comment{we will use GAE\cite{schulman2015high}}
      \State Compute policy update
      \begin{equation*}
        \mathbf{\theta}_{k+1} = arg \max_\theta \mathcal{L}^{CLIP}_{\theta_k}(\theta)
      \end{equation*}
      \State by taking K steps of minibatch SGD (via Adam \cite{kingma2014adam}), where
      \begin{equation*}
        \mathcal{L}^{CLIP}_{\theta_k}(\theta) =  E_t \left[ \sum_{t=0}^T\left[  min(r_t(\theta)\hat A^{\pi_k}_t \,, \, clip(r_t(\theta) \,,\, 1-\epsilon\,,\, 1+ \epsilon)\hat A_t^{\pi_k})\right]\right]
      \end{equation*}
    \EndFor
  \end{algorithmic}
\end{algorithm}



\begin{comment}

One of the most often used gradient estimator uses the following formula:

\begin{equation}
\hat g = \hat E_t \left[ \nabla_\theta log \pi_\theta ( a_t | s_t) \hat J_t  \right]
\end{equation}

Here the expected value is taken over a minibatch in an algorithm that alternates sampling and optimization.

  \bigskip


\begin{algorithm}
  \caption{basic PPO structure}\label{alg:cap}
  \begin{algorithmic}
      \For{iteration = 1,2 ...}
      \For{actor = 1,2,..., N}
        \State Run policy \( \pi_{\theta_{old}} \) in enviroment for \( T \) time-steps
        \State Compute advantage estimate \( \hat A_1 , ... , \hat A_T \)
      \EndFor
      \State Opt. surrogate \( L \) wrt \( \theta \), with K epochs and minibatch size \( M \leq NT \)
      \State $\theta_{old} \gets \theta$
    \EndFor
  \end{algorithmic}
\end{algorithm}

\end{comment}




 
%------------------------------------------------------------------------------------




%------------------------------------------------------------------------------------
% CHAPTER 3
%------------------------------------------------------------------------------------
%\chapter{Optimal taxation}
%\textcolor{red}{ future work}

\begin{itemize}
    \item Simualting the concept of tax evasion
    \item simulating coexisting tax and migration of high skilled agents
    \item simualtion with RL policymaker
    \item 
\end{itemize}
%------------------------------------------------------------------------------------



%------------------------------------------------------------------------------------
% CHAPTER 4 
%------------------------------------------------------------------------------------
\chapter{Experiments}
In this chapter I will present the two phases learning process utilized by the fully connected neural network (FCNet) to maximize the agents utility. Afterwards there will be a brief description of the 5 environments I am going to study (one per taxation analyzed) and my conjecture on how the agents will respond to these taxation.


\section{Experiment Setup}

After understanding the Foundation framework and the process used by RL to optimize a policy we can focus on the model used. In the paper used as a reference the authors used a combination of a convolutional neural network and a LSTM \cite{zheng2020ai}. This could grant them good spatial information from the CNN and a memory buffer from the LSTM. These feature are sensate because there is the need of process a map and because the agents share the same network.

In the experiments, however, it was used a fully connected neural network that do not share the same feature as the one above. The complete stucture of the model is shown in Table \ref{tab:Fcnet}. 


\begin{table}[h!]
    \begin{tabular}{llll}
	Model: "FCNet"            &                    &          &                            \\ \hline
	Layer (type)              & Output Shape       & Param \# & Connected to               \\ \hline
	observations (InputLayer) & {[}(None, 1260){]} & 0        &                            \\ \hline
	encoder (Encoder)         & (None, 256)        & 389632   & observations{[}0{]}{[}0{]} \\ \hline
	fc\_out (Dense)           & (None, 50)         & 12850    & encoder{[}0{]}{[}0{]}      \\ \hline
	value\_out (Dense)        & (None, 1)          & 257      & encoder{[}0{]}{[}0{]}      \\ \hline
	Total params: 402,739     &                    &          &                            \\
	Trainable params: 402,739 &                    &          &                            \\
	Non-trainable params: 0   &                    &          &                            \\ \hline
    \end{tabular}
    \caption{\label{tab:Fcnet} Fully connected neural network utilized.}
\end{table}


This setup might encur in a loss of efficiency for plenty of reasons. First of all, as said before, the feature of the LSTM/CNN are missing. Then due to lack of resources it was not possible to do a proper hyper parameter tuning, probably leading to inefficiencies. However, as we will see, the FCNet can provide us with meaningful results, in particular the agents will show emerging behaviors and specialization. These are feature that were also finded in the paper by Zhang and Trott, the difference is present in the efficiency of the agents that is lower for the model used here.

There are many parameters that have to be set at this stage (for a semi-complete list check the appendix \textcolor{red}{appendice}). A first part of these values are related to the training, while a second part to the environment. I would like to stress some of the latter parameters, to better undersand the simulation.

First the skill levels, these are set to be different for each agent, the skill is the amount of coins received when building an house. The set of 4 skills is fixed and these are taken from a pareto distribution. Namely the values are (11.3; 13.3; 16.5; 22.2). These value are always assinged to the same starting location in the map, thus the agent starting in the bottom right will always have the highest skill and so on. Other important parameters are the episode lenght \( H \) which is set to 1000 time steps, the resource respawn probability which is set to 1\% per timestep and the inital coin endowment that is set to 0.

Once defined the environment is defined it is possible to start the training, this will happen in two phases. The first phase to introduce labour, and the second to introduce the taxation.

\section{Phase 1 training}

The first training phase is necessary in order to get the FCNet \ref{tab:Fcnet} accustomed to the world dynamics and avoid falling into unoptimal behavior once disincentives factors are introduced. These two factors, as already discussed, are labour costs and taxation. In this first phase the interest is in getting the agent accustomed to the labour costs, while the taxes are completely removed and will be investigated in the second phase of training. 


\begin{figure}[h!]
    \centering
    \linespread{.9} 
    \includegraphics[width=0.95\textwidth]{Resources/imgs/LR_phase1.png}
    \caption[Learning rate and Labor weighed cost for Phase 1 training: ]%
    {\label{img:lr_phase0}Learning rate and Labor weighed cost for Phase 1 training: \small \textit{this was the schedule for respectively the learning rate and the labor cost in the phase 1 of the training, here the learning rate goes to 0 because we don't seek convergence now.}}
\end{figure}

The training agenda is built to run for a total of 30M time-steps, as we can see from Figure \ref{img:lr_phase0} the learning rate is set to be at 1e-4 for 10M steps then it linearly reduces to 0 in 20M time-steps. While the labour weight increases following the function:

\begin{equation}
    \vartheta_k = 1- exp\left(- \frac{\textit{episode completitions}}{\textit{energy warm-up constant (k)}}\right)
\end{equation}

where the warm-up constant is set to 10000. This setup gives the FCNet time to learn how to respond properly to the dis-utility generated by the labor. 



\begin{figure}[h!]
    \centering
    \linespread{.9} 
    \includegraphics[width=0.95\textwidth]{Resources/imgs/FCNET_prod_phase1.png}
    \caption[Equality and Productivity during Phase 1:  ]%
    {\label{img:prod_phase0}Equality and Productivity during Phase 1: \small \textit{here we have the evolution of equality between agents and total production in the training of phase 1}}
\end{figure}


In Figure \ref{img:prod_phase0} we can see two of the most important variables that we are considering. On the left we have the average equality, this value is calculated by the following formula:

\begin{equation}
    equality(\mathbf{x^c_{t}}) = 1 - \frac{N}{N-1} \frac{\sum^N_{i=1}\sum^N_{j=1}| x^c_{i,t} - x^c_{j,t}|}{2N\sum^N_{i=1}x^c_{i,t}}
\end{equation}

with \( 0 < equality(\mathbf{x^c_{t}}) < 1 \), this function returns 1 if the endowments are equally split between the N (4) agents, and 0 if only one agent owns all of the coin in the economy. What we see is that after 36M time-steps the equality settled around 0.73, this is justified by the difference in coin endowment between the agent with the highest skill (Agent 0 in Figure \ref{img:p0_brakedown}) and the other agents. And we see that the average productivity of he economy settles around 1400. 


\begin{figure}[h!]
    \centering
    \includegraphics[width=0.95\textwidth]{Resources/imgs/Figure_1.png}
    \includegraphics[width=0.80\textwidth]{Resources/imgs/Figure_2.png}
    \includegraphics[width=0.80\textwidth]{Resources/imgs/Figure_3.png}
    \caption[Phase 1 full brake down: ]%
    {\label{img:p0_brakedown}Phase 1 full brake down: \small \textit{this is a more complete overview of one random simulation crated using the weights at the end of phase 1 training}}
\end{figure}


The results obtained so far are not important for the final analysis since there were no criterion used to stop the training other than a time constraint. This means that the FCNet might have not reached convergence and an optimal policy. However we can keep these results in mind as a benchmark for the phase 2 trainig as we will use the parameters obtained to carry on the training.


\section{Phase 2 Trainig}

    
In the second phase of the training the learning rate schedule is changed to be 3e-4 for the first 35M time-steps and then will linearly decrease to 1e-6 in a span of 15M time-steps. This will grant a faster learning rate in the inital part of the trainig when the FCNet will suffer a huge decrease in efficiency due to the introduction of the taxation. 

The starting point for this second phase are the weights obtained from phase 1, and the training will be formed of five different simulations diverging from this inital state. These simulation are: US taxation, Italian Taxation, free market, Communism and Flat Tax.

\subsection{Free Trade}

Free trade is the first and simplest kind of simulation that we can carry on starting from the result in phase 1. This training consist in keep training the FCNet without imposing any kind of taxation. 

It is reasonable to belive that the result of this first training should produce a more productive economy since the agent's utility function is positive in marginal coin endowment. However the equality between agents might suffer from a higher production.

\subsection{US taxation}

The second kind of taxation that will is used is the US taxation. This is based on the 2018 Federal brachets and percentages. In particular we can see from Table \ref{tab:us_tax} the detailed stucture. 

\begin{table}[h!]
\begin{adjustbox}{max width=\textwidth}
    \begin{tabular}{llllllll}
        \hline
    Bracket in \$   & 0-9700 & 9700-39475 & 39475-84200 & 84200-160725 & 160725-204100 & 204100-510300 & 510300+ \\
    Bracket in Coin & 0-9.7  & 9.7-39.5   & 39.5-84.2   & 84.2-160.7   & 160.7-204.1   & 204.1-510.3   & 510.3+  \\
    Tax*            & 10\%   & 12\%       & 22\%        & 24\%         & 32\%          & 35\%          & 37\%\\
    \hline
    \end{tabular}
\end{adjustbox}
    \caption[2018 US federal tax system:]%
    {\label{tab:us_tax}2018 US federal tax system: \small \textit{this is the proportional system that was in place in the US in 2018, notice that the tax is marginal (i.e. if you are in the second bracket you will pay 970\$ + 12\% of the amount above 9700\$) }}
\end{table}

With the introduction of these lump sum tax plus redistribution is easy to conclude that there might me an increase in equality, however this migh produce a negative effect on total production since the high-skilled agent are disincentivised to produce.

\subsection{Italian Taxation}

As of the US's the Italian is a marginal system with brackets, however in this case the fiscal pressure is higer and the brachets smaller. As a reference it was used the 2020 INPS brachets and percentages (Table \ref{tab:ita_tax}). However, the brackets were calculated as if in dollars, so all the values were multiplied by 1.1575, the exchange rate at October 1st 2021.


\begin{table}[h!]
    \begin{adjustbox}{max width=\textwidth}
        \begin{tabular}{llllll}
            \hline
        Bracket in \EURcr   & 0-15000 & 15000-28000 & 28000-55000 & 55000-75000 & 75000+ \\
        Bracket in Coin & 0-17  & 17-32  & 32-63  & 63-86   &  86+   \\
        Tax*            & 23\%   & 27\%       & 38\%        & 41\%         & 43\%       \\
        \hline
        \end{tabular}
    \end{adjustbox}
        \caption[2018 US federal tax system:]%
        {\label{tab:ita_tax}2020 INPS tax system: \small \textit{this is the proportional system that was in place in the Italy in 2020, notice that the tax is marginal (i.e. if you are in the second bracket you will pay 3450\EURcr + 27\% of the amount above 15000\EURcr) }}
    \end{table}



In this case there is an even higher pressure, wich might push toward higher equality and lower total production.


\subsection{Communism}

Communism is the opposite to Free market, in this simulation the fiscal pressure is 1, meaning that every 100 timestep all the agents are taxed for the entire amount of coins they own and then redistibuted. This system will probably deliver the loewst production and the highest equality.

\subsection{Flat tax}



\section{Training Results}


\begin{figure}[h!]
    \centering
    \includegraphics[width=0.95\textwidth]{Resources/imgs/equality_training.png}
    \includegraphics[width=0.95\textwidth]{Resources/imgs/productivity_training.png}
    \includegraphics[width=0.95\textwidth]{Resources/imgs/utility_training.png}
    \caption[Phase 1 full brake down: ]%
    {\label{img:p1_brakedown}Phase 2 full brake down: \small \textit{this is a more complete overview of one random simulation crated using the weights at the end of phase 1 training}}
\end{figure}

%------------------------------------------------------------------------------------

%------------------------------------------------------------------------------------
% CHAPTER 5
%------------------------------------------------------------------------------------
\addcontentsline{toc}{chapter}{Conclusions}
\chapter*{Conclusions}

We have seen how reinforcement learning could be used to produce economical simulations. In detail, we saw RL applied to the gather-and-trade setup created by Alex Trott and Stephan Zheng. Here, we noticed how it is possible to maximize agents' utility through proximal policy optimization, a kind of gradient descent optimization. And we saw how this gave birth to agents' specialization, a feature that is hard to capture analytically. This first result is noticeable since is a proof of concept that is possible to create an experimental setup that is, once trained, easy to interact with and easy to run multiple times to gather data.

Afterward, we saw four different taxation systems (and redistribution) applied to a common starting point and commented on the difference in productivity and equality. When we compared the reproduction results, we noticed a loss in efficiency. This was justified from the model utilized and from the lack of hyperparameter tuning. Nonetheless, we reasonably found out, that there is a negative impact of taxation on production and a positive impact on equality. This impact cannot be properly quantified as a partial effect for a multitude of reasons. At first, there is the need for an efficient model, then, we need enough data to draw statistical conclusions. 


\subsection*{Future work}

This simulation can be a nice starting point for a multitude of future works. At first, is possible to study the application of RL in a simpler setup that can be modeled in an analytical way and check if RL converges to the analytical solutions. Afterward, starting from this model is possible to explore plenty of scenarios. For example, the brain drain and tax evasion could be interesting to model. For the former, we could simulate two countries that coexist and we could give the agents the choice to move from country A to country B. Then we could study how the high and low skilled agents decide to move across the countries. For the latter is possible to add the choice to reveal the real income and a small possibility of being caught. Here the interest would be in if or how the agents decide to evade given different tax systems. 

These are just two examples, but the freedom of design is technically endless. Of course, such research would need better machines, thus computational power, than the one's used for this thesis.
%------------------------------------------------------------------------------------







\renewcommand{\chaptermark}[1]{\markright{\thechapter \ #1}{}}
\lhead[\fancyplain{}{\bfseries\thepage}]{\fancyplain{}{\bfseries\rightmark}}
\appendix   
%\addcontentsline{toc}{chapter}{Appendix}
\chapter{Tables, Algorithms and Graphs}  



\begin{table}[]
    \centering
    \begin{tabular}{lll}
    \hline
    Parameter                    &  & Value \\ \hline
    Episode Lenght               & \( H \) & 1000  \\
    World height                 &  & 25    \\
    World width                  &  & 25    \\
    Resources respawn prob.      &  & 0.01  \\
    Inital Coin endowment        &  & 0     \\
    Iso-elastic utility exponent &  & 0.23  \\
    Move labor                   &  & 0.21  \\
    Gather labor                 &  & 0.21  \\
    Trade labor                  &  & 0.05  \\
    Build labor                  &  & 2.1   \\
    Base build payout            &  & 10    \\
    Max skill multiplier         &  & 3     \\
    Max bid/ask price            &  & 10    \\
    Max bid/ask order duration   &  & 50    \\
    Max simultaneous open orders &  & 5     \\
    Tax period duration          &  & 100   \\
    Min bracket rate             &  & 0\%   \\
    Max bracket rate             &  & 100\% \\ \hline
    \end{tabular}
    \caption{\label{tab:hyperparameter_env}Your caption.}
\end{table}


\begin{table}[]
    \centering
    \begin{tabular}{lll}
    \hline
    Variable Name                                 & Dimension   & Bounds                      \\ \hline
    world Map                                     & (7, 11, 11) & \{0;1\}                     \\ 
    world-idx\_map                                & (2, 11, 11) & \{0,1,...,5\}               \\ 
    world-loc-row                                 & (1,)        & {[}0,1{]}                   \\ 
    world-loc-col                                 & (1,)        & {[}0,1{]}                   \\ 
    world-inventory-Coin                          & (1,)        & {[}0,inf)                   \\ 
    world-inventory-Stone                         & (1,)        &                             \\ 
    world-inventory-Wood                          & (1,)        &                             \\ 
    time                                          & (1, 1)      &                             \\ 
    Build-build\_payment                          & (1,)        &                             \\ 
    Build-build\_skill                            & (1,)        &                             \\ 
    ContinuousDoubleAuction-market\_rate-Stone    & (1,)        &                             \\ 
    ContinuousDoubleAuction-price\_history-Stone  & (11,)       &                             \\ 
    ContinuousDoubleAuction-available\_asks-Stone & (11,)       &                             \\ 
    ContinuousDoubleAuction-available\_bids-Stone & (11,)       &                             \\ 
    ContinuousDoubleAuction-my\_asks-Stone        & (11,)       &                             \\ 
    ContinuousDoubleAuction-my\_bids-Stone        & (11,)       &                             \\ 
    ContinuousDoubleAuction-market\_rate-Wood     & (1,)        &                             \\ 
    ContinuousDoubleAuction-price\_history-Wood   & (11,)       &                             \\ 
    ContinuousDoubleAuction-available\_asks-Wood  & (11,)       &                             \\ 
    ContinuousDoubleAuction-available\_bids-Wood  & (11,)       &                             \\ 
    ContinuousDoubleAuction-my\_asks-Wood         & (11,)       &                             \\ 
    ContinuousDoubleAuction-my\_bids-Wood         & (11,)       &                             \\ 
    Gather-bonus\_gather\_prob                    & (1,)        &                             \\ 
    action\_mask                                  & (50,)       &                             \\ \hline
    \end{tabular}
    \caption{\label{tab:full_obs} Full observation space.}
\end{table}


\rhead[\fancyplain{}{\bfseries \thechapter \:Tables}]
{\fancyplain{}{\bfseries\thepage}}


\nocite{ai-economist}
\nocite{aie_git}
\nocite{akira_git}
\nocite{akira_medium}
\nocite{my_git}

\printbibliography[type=article]
\printbibliography[type=online, heading=subbibliography, title=Sitography]
\printbibliography[type=online2, heading=subbibliography, title=Source Code]
\end{document}
