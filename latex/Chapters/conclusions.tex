
We have seen how reinforcement learning could be used to produce economical simulations. In detail, we saw RL applied to the gather-and-trade setup created by Alex Trott and Stephan Zheng. Here, we noticed how it is possible to maximize agents' utility through proximal policy optimization, a kind of gradient descent optimization. And we saw how this gave birth to agents' specialization, a feature that is hard to capture analytically. This first result is noticeable since is a proof of concept that is possible to create an experimental setup that is, once trained, easy to interact with and easy to run multiple times to gather data.

Afterward, we saw four different taxation systems (and redistribution) applied to a common starting point and commented on the difference in productivity and equality. When we compared the reproduction results, we noticed a loss in efficiency. This was justified from the model utilized and from the lack of hyperparameter tuning. Nonetheless, we reasonably found out, that there is a negative impact of taxation on production and a positive impact on equality. This impact cannot be properly quantified as a partial effect for a multitude of reasons. At first, there is the need for an efficient model, then, we need enough data to draw statistical conclusions. 


\subsection*{Future work}

This simulation can be a nice starting point for a multitude of future works. At first, is possible to study the application of RL in a simpler setup that can be modeled in an analytical way and check if RL converges to the analytical solutions. Afterward, starting from this model is possible to explore plenty of scenarios. For example, the brain drain and tax evasion could be interesting to model. For the former, we could simulate two countries that coexist and we could give the agents the choice to move from country A to country B. Then we could study how the high and low skilled agents decide to move across the countries. For the latter is possible to add the choice to reveal the real income and a small possibility of being caught. Here the interest would be in if or how the agents decide to evade given different tax systems. 

These are just two examples, but the freedom of design is technically endless. Of course, such research would need better machines, thus computational power, than the one's used for this thesis.