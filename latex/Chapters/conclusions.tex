
We have seen how reinforcement learning could be used to produce economical simulations. In detail, we saw RL applied to the gather-and-trade setup created by Alex Trott and Stephan Zheng. Here, we noticed how it is possible to maximize agents utility through proximal policy optimization, a kind of gradient descent optimization. And we saw how this gave birth to agents specialization, a feature that is hard to capture analytically. Afterwards we saw four different taxation system (and redistribution) applied to a common starting point and commented on the difference on productivity and equality. We found out, reasonably, that there is a negative impact of taxation on production and a positive impact on equality. This impact cannot be properly quantified as partial effect for a multitude of reasons. However it was possible to compare the results in performance terms. 




\subsection*{Future work}

This simulations can be a nice starting point for a multitude of future works. At first is possible to study the application of RL in simpler setup that can completely modeled in an analytical way and check if RL converges to the analytical solutions. Afterward, starting from this model, is possible to explore plenty scenarios. A couple that could be interesting in my opinion are the brain drain and tax evasion. For the former we could simulate two countries that coexist and we could give the agents the choice to move from country A to country B. Then we could study how the high and low skilled agents decide to move across the countries. For the latter is possible to add the choice to reveal the real income and a small possibility to being caught. Here the interest would be in if or how the agents decide to evade given different tax systems. 

These are just two examples, but the freedom of design is technically endless. Of course such research would need better machines, thus computational power, than the one's used for this thesis.